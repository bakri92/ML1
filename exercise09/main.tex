\documentclass[11pt,abstract=on]{scrartcl}
\usepackage{graphicx}
\usepackage{graphics}
\usepackage[utf8]{inputenc}
\usepackage[T1]{fontenc}
\usepackage[ngerman]{babel}
\usepackage{svg}
\usepackage{amsmath}
\usepackage{import}
\usepackage{array}
\usepackage{longtable}
\usepackage[nolist]{acronym} %[nolist]
%\usepackage{hyphenat}
\title{Excercise Sheet 9}
\subtitle{Machine Learning}
\date{\today}
\setlength{\parindent}{0pt}
\begin{document}
\maketitle
\section{The Dual SVM}
\subsection{a}
\begin{equation*}
\Lambda(w,\theta,\alpha) = \frac{1}{2}  \lVert w \rVert^2 - \sum\limits_{i=1}^{N} \alpha_{i}(y_{i}(w^T x_{i} + \theta)-1)
\end{equation*}

\subsection{b}
Derive $\Lambda$ by w and $\theta$ and set the derivatives to zero

\begin{equation*}
\frac{\partial}{\partial w} \Lambda = 0 \Rightarrow w = \sum\limits_{i=1}^{N}\alpha_i y_i x_i
\end{equation*}
\begin{equation*}
 \frac{\partial}{\partial \theta} \Lambda = 0 \Rightarrow \sum\limits_{i=1}^{N}\alpha_i y_i = 0
\end{equation*}

The function W($\alpha$) is determined by inserting the results into $\Lambda$

\begin{equation*}
W(\alpha) = \frac{1}{2}\sum\limits_{i=1}^{N}(a_i y_i x_i)^2 - \sum\limits_{i=1}^{N} \alpha_i y_i x_i \sum\limits_{i=1}^{N} \alpha_i y_i x_i - \theta \sum\limits_{i=1}^{N} \alpha_i y_i + \sum\limits_{i=1}^{N} \alpha_i
\end{equation*}
\begin{equation*}
= \sum\limits_{i=1}^{N} \alpha_i-\frac{1}{2} \sum\limits_{i=1}^{N} \alpha_i y_i x_i \sum\limits_{i=1}^{N} \alpha_i y_i x_i
\end{equation*}
\begin{equation*}
= \sum\limits_{i=1}^{N} \alpha_i-\frac{1}{2} \sum\limits_{i,j=1}^{N} \alpha_i \alpha_j y_i y_j x_i x_j
\end{equation*}

The function W($\alpha$) is maximized by $\alpha$. For each $x_i$ and $y_i$ in the training data, $\alpha_i$ depends on

\begin{equation*}
y_i((w x_i) + \theta) > 1 \rightarrow \alpha_i = 0
\end{equation*}
\begin{equation*}
y_i((w x_i) + \theta) = 1 \rightarrow \alpha_i \in SV
\end{equation*}

The vector w only depends on support vectors

\begin{equation*}
w = \sum\limits_{i=1}^{N} \alpha_i y_i x_i = \sum\limits_{i=1}^{\#SV} \alpha_i y_i x_i
\end{equation*}

\begin{equation*}
y_i((\sum\limits_{j=1}^{\#SV} \alpha_j y_j x_j x_i) + \theta) > 1 \rightarrow \alpha_i = 0
\end{equation*}
\begin{equation*}
y_i((\sum\limits_{j=1}^{\#SV} \alpha_j y_j x_j x_i) + \theta) = 1 \rightarrow \alpha_i \in SV
\end{equation*}

The value of $\theta$ can be recovered with an $x_m$ on the margin with

\begin{equation*}
y_m (w^T x_m + \theta) = 1 \Leftrightarrow \theta = y_m - w x_m
\end{equation*}

\subsection{c}


Derive $\Lambda$ by w and $\theta$ and set the derivatives to zero

\begin{equation*}
\frac{\partial}{\partial w} \Lambda = 0 \Rightarrow w = \sum\limits_{i=1}^{N}\alpha_i y_i \Phi(x_i)
\end{equation*}
\begin{equation*}
 \frac{\partial}{\partial \theta} \Lambda = 0 \Rightarrow \sum\limits_{i=1}^{N}\alpha_i y_i = 0
\end{equation*}

The function W($\alpha$) is determined by inserting the results into $\Lambda$

\begin{equation*}
W(\alpha) = \frac{1}{2}\sum\limits_{i=1}^{N}(a_i y_i \Phi(x_i))^2 - \sum\limits_{i=1}^{N} \alpha_i y_i \Phi(x_i) \sum\limits_{i=1}^{N} \alpha_i y_i \Phi(x_i) - \theta \sum\limits_{i=1}^{N} \alpha_i y_i + \sum\limits_{i=1}^{N} \alpha_i
\end{equation*}
\begin{equation*}
= \sum\limits_{i=1}^{N} \alpha_i-\frac{1}{2} \sum\limits_{i=1}^{N} \alpha_i y_i \Phi(x_i) \sum\limits_{i=1}^{N} \alpha_i y_i \Phi(x_i)
\end{equation*}
\begin{equation*}
= \sum\limits_{i=1}^{N} \alpha_i-\frac{1}{2} \sum\limits_{i,j=1}^{N} \alpha_i \alpha_j y_i y_j k(x_i, x_j)
\end{equation*}

The function W($\alpha$) is maximized by $\alpha$. For each $x_i$ and $y_i$ in the training data, $\alpha_i$ depends on

\begin{equation*}
y_i((w \Phi(x_i)) + \theta) > 1 \rightarrow \alpha_i = 0
\end{equation*}
\begin{equation*}
y_i((w \Phi(x_i)) + \theta) = 1 \rightarrow \alpha_i \in SV
\end{equation*}

The vector w only depends on support vectors

\begin{equation*}
w = \sum\limits_{i=1}^{N} \alpha_i y_i \Phi(x_i) = \sum\limits_{i=1}^{\#SV} \alpha_i y_i \Phi(x_i)
\end{equation*}

\begin{equation*}
y_i((\sum\limits_{j=1}^{\#SV} \alpha_j y_j k(x_j, x_i)) + \theta) > 1 \rightarrow \alpha_i = 0
\end{equation*}
\begin{equation*}
y_i((\sum\limits_{j=1}^{\#SV} \alpha_j y_j k(x_j, x_i)) + \theta) = 1 \rightarrow \alpha_i \in SV
\end{equation*}

The value of $\theta$ can be recovered with an $x_m$ on the margin with

\begin{equation*}
y_m (w^T \Phi(x_m) + \theta) = 1 \Leftrightarrow \theta = y_m - w \Phi(x_m)
\end{equation*}

\section{SVMs and Quadratic Programming}
\begin{equation*}
x= \alpha
\end{equation*}
\begin{equation*}
P=H
\end{equation*}
\begin{equation*}
q=-\boldsymbol{1}
\end{equation*}
\begin{equation*}
G=\left( \begin{array}{rrrr}
1 & 0 & \cdots & 0 \\
0 & \ddots & 0 & \vdots \\
\vdots & 0 & \ddots & 0 \\
0 & \cdots & 0 & 1 \\
-1 & 0 & \cdots & 0 \\
0 & \ddots & 0 & \vdots \\
\vdots & 0 & \ddots & 0 \\
0 & \cdots & 0 & -1 \\
\end{array}\right), 2N\times N 
\end{equation*}
\begin{equation*}
 h = (C,\cdots,C,0,\cdots,0)^T,2N
\end{equation*}
\begin{equation*}
A=\boldsymbol{y}^T
\end{equation*}
\begin{equation*}
b = \boldsymbol{0}
\end{equation*}


\end{document}
